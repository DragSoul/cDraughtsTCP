%pdflatex -halt-on-error -aux-directory=tmp -output-directory=tmp rapport.tex%

\documentclass{article}
\usepackage{amsmath}
\usepackage[utf8]{inputenc}
\usepackage[T1]{fontenc}
\usepackage{graphicx}
\usepackage{hyperref}

\title{Rapport de projet : Jeu de Dames en réseau}
\author{Anthony Bertrand, Quentin Bandera}
\date{}

\begin{document}
    \pagenumbering{gobble}
    \maketitle
    \tableofcontents
    \newpage
    \pagenumbering{arabic}
    \section{Introduction}

    \section{notes}
        \subsection{choix du protocole de transport}
        Dans le monde du jeu vidéo en ligne, le protocole UDP est très souvent utilisé. UDP possède une faible latence pour le traitement
        ce qui est l'élement recherché dans des jeux nerveux comme Overwatch, League of Legends, Unreal Tournament etc... Dans notre
        cas, le jeu se fera en tour par tour. Notre choix se porte donc sur TCP car nous ne voulons pas de perte de paquet lorsque le joueur
        valide son coup.\\
        Nous aurions pu utiliser un protocole applicatif existant, se basant sur UDP, voire même créer notre propre protocole mais par soucis
        de temps, nous nous contenterons de TCP.
\end{document}
