%pdflatex -halt-on-error -aux-directory=tmp -output-directory=tmp rapport.tex%

\documentclass{article}
\usepackage{amsmath}
\usepackage[utf8]{inputenc}
\usepackage[T1]{fontenc}
\usepackage{graphicx}
\usepackage{hyperref}

\title{Rapport de projet : Jeu de Dames en réseau}
\author{Anthony Bertrand, Quentin Bandera}
\date{}

\begin{document}
    \pagenumbering{gobble}
    \maketitle
    \tableofcontents
    \newpage
    \pagenumbering{arabic}
    \section{Introduction}
        Lors de nos études en L3 Informatique, nous avons dû implémenter un jeu de Dame en réseau. Le but de ce projet
        est de mettre en pratique les connaissances acquises aussi bien en cours magistral qu'en travaux pratiques.
        Nous allons tout d'abord revenir sur les règles à implémenter pour le jeu de Dame. Nous verrons ensuite quels sont
        les choix à notre disposition en ce qui concerne les protocoles réseaux. Enfin, nous expliqueront les problèmes
        rencontrés ainsi que leur potentiel résolution.

    \section{Le jeu de Dame}
    \subsection{Origine}
    \subsection{Règles du jeux}

    \section{Travail effectué}
        Avant de commencé cette partie, nous tenons à préciser que ce projet a dû être réalisé en deux semaine, par des étudiants
        ayant deux mois d'expérience dans le domaine du réseau. Le temps étant la ressource qui nous fit le plus défaut, nous avons
        décidé de nous concentrer au maximum sur la partie réseau et non sur la partie algorithmique du jeu de Dame.
    \subsection{Choix du protocole réseau}
    \section{Notes}
        \subsection{choix du protocole de transport}
        Dans le monde du jeu vidéo en ligne, le protocole UDP est très souvent utilisé. UDP possède une faible latence pour le traitement
        ce qui est l'élement recherché dans des jeux nerveux comme Overwatch, League of Legends, Unreal Tournament etc... Dans notre
        cas, le jeu se fera en tour par tour. Notre choix se porte donc sur TCP car nous ne voulons pas de perte de paquet lorsque le joueur
        valide son coup.\\
        Nous aurions pu utiliser un protocole applicatif existant, se basant sur UDP, voire même créer notre propre protocole mais par soucis
        de temps, nous nous contenterons de TCP.
\end{document}
